% This is "sig-alternate.tex" V2.0 May 2012
% This file should be compiled with V2.5 of "sig-alternate.cls" May 2012
%
% This example file demonstrates the use of the 'sig-alternate.cls'
% V2.5 LaTeX2e document class file. It is for those submitting
% articles to ACM Conference Proceedings WHO DO NOT WISH TO
% STRICTLY ADHERE TO THE SIGS (PUBS-BOARD-ENDORSED) STYLE.
% The 'sig-alternate.cls' file will produce a similar-looking,
% albeit, 'tighter' paper resulting in, invariably, fewer pages.
%
% ----------------------------------------------------------------------------------------------------------------
% This .tex file (and associated .cls V2.5) produces:
%       1) The Permission Statement
%       2) The Conference (location) Info information
%       3) The Copyright Line with ACM data
%       4) NO page numbers
%
% as against the acm_proc_article-sp.cls file which
% DOES NOT produce 1) thru' 3) above.
%
% Using 'sig-alternate.cls' you have control, however, from within
% the source .tex file, over both the CopyrightYear
% (defaulted to 200X) and the ACM Copyright Data
% (defaulted to X-XXXXX-XX-X/XX/XX).
% e.g.
% \CopyrightYear{2007} will cause 2007 to appear in the copyright line.
% \crdata{0-12345-67-8/90/12} will cause 0-12345-67-8/90/12 to appear in the copyright line.
%
% ---------------------------------------------------------------------------------------------------------------
% This .tex source is an example which *does* use
% the .bib file (from which the .bbl file % is produced).
% REMEMBER HOWEVER: After having produced the .bbl file,
% and prior to final submission, you *NEED* to 'insert'
% your .bbl file into your source .tex file so as to provide
% ONE 'self-contained' source file.
%
% ================= IF YOU HAVE QUESTIONS =======================
% Questions regarding the SIGS styles, SIGS policies and
% procedures, Conferences etc. should be sent to
% Adrienne Griscti (griscti@acm.org)
%
% Technical questions _only_ to
% Gerald Murray (murray@hq.acm.org)
% ===============================================================
%
% For tracking purposes - this is V2.0 - May 2012

\documentclass{e511-project}

\begin{document}
%
% --- Author Metadata here ---
%\CopyrightYear{2007} % Allows default copyright year (20XX) to be over-ridden - IF NEED BE.
%\crdata{0-12345-67-8/90/01}  % Allows default copyright data (0-89791-88-6/97/05) to be over-ridden - IF NEED BE.
% --- End of Author Metadata ---

\title{Project Proposal: Monaural Single-Source Speaker Separation}
\subtitle{E511 Machine Language for Signal Processing}
%
% You need the command \numberofauthors to handle the 'placement
% and alignment' of the authors beneath the title.
%
% For aesthetic reasons, we recommend 'three authors at a time'
% i.e. three 'name/affiliation blocks' be placed beneath the title.
%
% NOTE: You are NOT restricted in how many 'rows' of
% "name/affiliations" may appear. We just ask that you restrict
% the number of 'columns' to three.
%
% Because of the available 'opening page real-estate'
% we ask you to refrain from putting more than six authors
% (two rows with three columns) beneath the article title.
% More than six makes the first-page appear very cluttered indeed.
%
% Use the \alignauthor commands to handle the names
% and affiliations for an 'aesthetic maximum' of six authors.
% Add names, affiliations, addresses for
% the seventh etc. author(s) as the argument for the
% \additionalauthors command.
% These 'additional authors' will be output/set for you
% without further effort on your part as the last section in
% the body of your article BEFORE References or any Appendices.

\numberofauthors{3} %  in this sample file, there are a *total*
% of EIGHT authors. SIX appear on the 'first-page' (for formatting
% reasons) and the remaining two appear in the \additionalauthors section.
%
\author{
% You can go ahead and credit any number of authors here,
% e.g. one 'row of three' or two rows (consisting of one row of three
% and a second row of one, two or three).
%
% The command \alignauthor (no curly braces needed) should
% precede each author name, affiliation/snail-mail address and
% e-mail address. Additionally, tag each line of
% affiliation/address with \affaddr, and tag the
% e-mail address with \email.
%
% 1st. author
\alignauthor
Jonathan Branam
       \email{jobranam@iu.edu}
% 2nd. author
\alignauthor
Jamie Israel
       \email{jgisrael@iu.edu}
% 3rd. author
\alignauthor
Hellen Jiang
       \email{heljiang@iu.edu}
}

\maketitle
\begin{abstract}
There have been several recent developments in the use of deep
learning to perform speaker diarisation, or separation of multiple
speech signals from a single source into separate homogenous signals
representing each individual speaker.  This task is particularly
challenging where the input is monophonic, there is no control over
environmental conditions, the number of speakers is unknown and there
are no samples of the speakers from which to train. One such
real-world application is for a call center seeking to analyze the
speech of agents and customers where the only source is a monophonic
recording of a call with unknown participants.  The center would like
to transcribe speech to text for content analysis, but they have been
unable to obtain an accurate translation due in large part to overlap
in the recorded signals (i.e., cross-talk) and the inability to
associate utterances with speakers.

Our initial step will be to apply the non-negative matrix
factorization technique illustrated in the class materials to perform
a baseline for this source separation problem. From there, we intend
to explore a neural network model using PyTorch to implement
Permutation Invariant Training (PIT) from
\cite{DBLP:journals/corr/KolbaekYTJ17} or utterance-level Permutation
Invariant Training (uPIT) from
\cite{DBLP:journals/corr/YuKTJ16}
as these seem to be the latest approaches to
single source speaker separation in the academic literature. The call
center recordings are not currently labeled, so we propose creating
our own test and training dataset by mixing single-speaker recordings
from either the TIMIT \cite{garofolo:timit} or WSJ0 
\cite{garofolo:wsj0} recordings.
The literature mentions the WSJ0-2mix data set that was used in
\cite{DBLP:journals/corr/HersheyCRW15} the seminal Deep Clustering
(DPCL) paper but we have yet to find a detailed specification for how to
generate this dataset.



\end{abstract}

%
% The following two commands are all you need in the
% initial runs of your .tex file to
% produce the bibliography for the citations in your paper.
\bibliographystyle{abbrv}
\bibliography{e511-proposal}
% You must have a proper ".bib" file
%  and remember to run:
% latex bibtex latex latex
% to resolve all references
%
% ACM needs 'a single self-contained file'!
%
\end{document}
